The software sector is a substantial contributor to the growth of the UK's economy. EngineeringUK reports: "\textit{software, IT and telecoms together generated 4.2\% of UK gross value added in 2011 and provided 885,000 jobs. There are 107,000 software businesses, and the UK is the world's number two exporter of telecoms services (£5.4 billion) and number three in computer services (£7.1 billion) and information services (£2 billion)"}.\footnote{\url{https://www.engineeringuk.com/media/1466/enguk-report-2015-interactive.pdf}}

The UK is already world-leading in software engineering. However, at
present there are no UK researchers active in the field of automated fix detection. Our project would put the UK at the forefront of such research. In addition, as 
software engineering is considered high priority growth areas by
EPSRC\@, our project will offer a bridgehead to work of national and international importance.

The proposed research will also produce new disruptive technologies, consistent with the EPSRC's deliver plan 17-2019/20. Automated fix detection aims to offload the laborious process of fix detection to an automatic process. Success in this area will thus challagen the status quo of software development. Developers will be able to spend more time on other tasks, making them more productive and subsequently facilitating further economic growth. 