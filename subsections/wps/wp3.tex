When a series of fixes have been integrated into the codebase, software managers want to know several additional statistics and features of the fixes to help them manage and understand their code. 
Following the work done in WP1 and WP2, we will provide sufficient statistical apparatus to present a time series describing the evolution of a piece of software over time in terms of its continual error causing propensity.  Investigating and developing these statistics and features will be the project of WP4. We divide the project into a quantitive sub-project, and qualitive sub-project, as follows.

To begin with, this will include finding appropriate statistical measures for quantities of interest to a software manager. Basic statistics which answer the following questions will be of value: how many fixes (or new bugs) are to be expected over a given time period, whats the average time it takes to create a fix, what's the expected likelihood a type of bug will recur, how many times has a fix been attempted for a given method, how many have worked, how stable is the code overall in terms of deviation from the mean. Accordingly, in the first part of the workpackage we will develop and investigate these measures. 

A second sub-problem involves correctly classifying a given fix. This involves i) correct bug classification (to know what bug was fixed), ii) the 
association of a fix for a bug, and iii) the type of fix repaired the bug.

WP1 and WP2 solve ii), and methods exist at bug classification to help solve i), and can involve test mining engineering reports, or testing of the program before and after the given fix. However, to our knowledge no work has been done at iii). Here, we (at least) want to know whether the fix fixed the bug in an efficient/inefficient, effective/ineffective, syntactically simple/complex way. For instance, a fix may have added an undue amount of runtime to a project, may have only been effective at solving a certain proportion of failures, and may involve a lot of complex code involving the addition of pointers or other new types of structures to the code. Any project manager will want to know the qualitive 

wrt ii) It is known bugs can be classified into many different classes. Common classes are environmental, syntax, and logical. In this WP, we thus need to explore taxonomies/ontologies of errors which are fixed, and the automatic methods to identify them. Thus far,methods exist at bug classification, and can involve test mining engineering reports, or testing of the program before and after the given fix, but not \textit{fix}-categorisation.  