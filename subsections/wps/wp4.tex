When a series of fixes have been integrated into the codebase, software managers want to know several additional things about their managed code. This includes answers to questions like, how stable is the development of the code, and what's the rate of improvement?
Following the work done in WP1 and WP2, we will have sufficient statistical apparatus to present a time series describing the evolution of a piece of software over time in terms of its continual error causing propensity. 

Accordingly, 
we are in a position to begin describing elementary statistics about the evolution of a software project. Investigating which of these statistics will be useful from the standpoint of a software manager, and developing these statistics, will be the first sub-project of WP4. 
This includes finding appropriate measures for rates of code improvement and code stability. 

A second sub-problem involves correctly classifying the problem that a given fix has fixed. This involves error classification. Errors can be classified into many different classes. Common classes are environmental, syntax, logical. In this WP, we thus need to explore taxonomies/ontologies of errors which are fixed, and automatic methods to identify what type of bug a fix has actually fixed. Solution to this problem can involve test mining engineering reports, or testing of the program before and after the given fix. 