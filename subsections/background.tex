Any solution to AFDP is expected to be a function of a failure's symptoms, such as stack traces and user/test reports. In large scale industrial projects, where software is developed using continual integration and deployment in conjunction with automatic verification and validation (V\&V), the need to solve this ``largely overlooked" problem is especially acute~\cite{Facebook1}. This is due to the existence of flaky tests, which are tests which could fail or pass for the same inputs over the program, but do not always indicate bugs in the code.
This problem of ADFP, in the light of flaky tests, has recently been decomposed by Alshahwan \etal  into two subproblems: 

\begin{itemize}
\item  \emph{Failure grouping}, which associates groups of failures to the methods which generate them, and 

\item \emph{Proving a negative}, which determines when we can be confident failures will not recur (i.e. a fix has succeeded). 
\end{itemize}

We take up the challenge of solving these sub-problems. To group failures, we investigate methods of causal inference to assign each failure a root cause (WP1).  To prove a negative, we will investigate statistical change point detection methods to detect when a fix has succeeded in the presence of flaky tests (WP2). Combined, these offer a novel solution to the fix detection problem which is at once scalable and integratable into Industrial development process (WP3). 

Finally, solving AFDP facilitates the solution of a related problem which we describe as finding useful \textit{Fix-classifications and statistics}. Solution to this is critical at the software managerial level for determining areas for software development in the short term furture, providing a high-level description of how and what fixed have occurred dynamically over time (WP4).  We decompose this problem into two sub-problems:

\begin{itemize}
\item \textit{Fix-classification}, where fixes are put into a class which describes the fix. This can include fixes which solve an environmental or a resource dependent problem.

\item \textit{Fix statistics}, which describe the evolution of fixes over time.
\end{itemize}

Solution to the four problems thus provides, for any large scale software project, answers to four respective questions: which fixes have fixed which failures, which methods can be said to be fixed, what class of problem did the fixes fix, and how have fixes evolved. This gives us a comprehensive and automated treatment for understanding the evolution of fixed code, which in an implemented tool, will provide critical high level understanding of code for program managers. 