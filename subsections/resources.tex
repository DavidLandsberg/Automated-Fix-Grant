The activies in the WPs will need to be closely integrated. The foundational work developed in each of the work packages
will strongly influence the outcomes of the other WPs. Given the novelty of the work, alternative solutions will be developed if others seem impractical
or unscaleable. 

We now discuss development issues. To develop WP1, we will need an experimental framework where we can evaluate the performance of different causal measures on given benchmarks using standard IR measures (such as accuracy, precision, recall, and F-scores). We will evaluate the measures on different testing scenarios which do not make many of the restrictive assumptions outlined in 2.1. For instance, if i) is not true we need to perform fault localisation using a causal measure on the updated program alone (using a given fault localisation setup~\cite{eval}). If ii) or iii) are not true we will need to employ measures empirically demonstrated to perform well in the presence of noise~\citep{pearl2016causal}. 

The development of WP2 will include an experimental comparison of different CDMs, testing for effectiveness and scalability when employed at the fix detection task. To measure effectiveness, we use standard IR methods~\citep{Aminikhanghahi, arXiv:1411.7955}. To measure scalability, we will measure practical runtime on representative benchmarks. This work is made feasible insofar as many CDMs are already implemented, known to scale well, and can be used in an "online" contexts involving continuous real-time streams of datapoints.\footnote{\scriptsize{\url{http://members.cbio.mines-paristech.fr/~thocking/change-tutorial/RK-CptWorkshop.html}}} 

The development of WP3 will include research into time series data analysis, and also methods used for stock market analysis. In an ideal case, we would expect a research output of a program analysis tool which could display time series analyses in the same way as a trading tool might. Here the data analysed would be in terms of code performance as opposed to stock performance.  Given the potential for modularity of such a tool, with many different stacked components, the development of this project would allow the production of anywhere between a lightweight and a heavyweight tool.

%% almost word for word from LUCID
\project{} will be based at UCL, London campus. Project leads will be EB and DL with an RA also
based at UCL\@. The RA's career development will be fostered through giving presentations,
organising workshops, and visiting industrial partners. The RA will be encouraged to present
work at internal seminars and international conferences. The RA will learn community and networking
skills by organising and chairing workshops at the host university on \project{} from the
prospective of expertise channels. 

%Visits to Microsoft will allow the establishment of personal and
%professional connections with industrial researchers. Visits to Chalmers in Sweden will foster
%personal and professional connections with language-based security researchers. 
