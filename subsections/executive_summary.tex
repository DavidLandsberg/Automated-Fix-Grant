
Software engineers spend most of their time maintaining, not writing, their programs. A critical part of maintenance involves discovering when a given program, previously demonstrated to be produce failures (violations of a given program specification), has been fixed. 
At present this is a manual process in which the program is 
inspected by the engineer and assessed as to whether a fix has occurred. Automation promises to improve the efficiency and effectiveness of this process, by delivering a report to the user which states that a method has been fixed with increased reliability, and in a smaller timescale than if performed manually. We call the problem of determining whether a fix has actually occurred in this way --- the \textit{automated fix detection problem} (AFDP). Solution to this problem 
has the potential to greatly improve the quality assurance of released software. It is this problem that our research will address. 
