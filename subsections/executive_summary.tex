
A critical element of software development involves knowing whether a given project, previously demonstrated to produce failures, has been fixed or not. Knowing this gives project managers the necessary quality assurance to make release decisions. 
At present, detecting whether a fix has occurred or not is a laborious manual process. Here, the engineer examines and tests the code to estimate whether a fix has occurred, the difficulty of which is exascerbated by the existence of flaky tests. Automation promises to improve the efficiency and effectiveness of this process, by delivering a report to the user which states that a method has been fixed with increased reliability, and in a smaller timescale than if performed manually. We call the problem of determining whether a fix has actually occurred in this way --- the \textit{automated fix detection problem} (AFDP). Solution to this problem has the potential to greatly improve the quality of released software. It is this problem that our research will address. 

%(violations of a given program specification)